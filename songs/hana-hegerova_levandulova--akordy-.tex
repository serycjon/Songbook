\beginsong{Levandulová (akordy)}[by={Hana Hegerová},
sr={},
cr={}]
\transpose{0}
% {{meta: source http://www.velkyzpevnik.cz/zpevnik/hana-hegerova/levandulova-akordy}}



\beginverse
1. Za\[D] mostem v ú\[Hmi]zké ulici
j\[F#mi]e krámek z dálk\[Hmi]y vonící
m\[D]eduňkou, rdesnem, sk\[Hmi]ořicí,
ale h\[Emi7]lavně, \[A]ach jo.
\endverse

\beginverse
L\[Emi7]éta snad z úcty k tr\[A]adici
k\[Emi7]upuji celou kr\[A]abici
a p\[Emi7]řes ní i papír b\[A]alicí,
cít\[Emi7]ím, že řekneš, \[A]ach jo.
\endverse


\beginchorus
R: Ty jsi m\[D]á \[Hmi]levandulov\[F#mi]á,
úplně c\[D]elá celičk\[Hmi]á, levandulov\[Emi7]á,
nádh\[Emi]erně \[A]levandulov\[Emi]á,
\[Emi7]hmm, \[A7]  levandulov\[D]á.\[Hmi]
\endchorus


\beginverse
2. Ne vždycky vařím dobroty
a občas mívám teploty
a nekapu ti do noty
a často vzdychám, ach jo.
Sotva však cinkneš za vraty,
otevřu, koukám, no a ty
upadáš vesměs v záchvaty
a vždycky voláš, ach jo.
\endverse


\beginchorus
R: Ty jsi má levandulová,
nádherná, celá levandulová,
famózně levandulová,
hmm, levandulová.
\endchorus


\beginverse
3. Už dobře pětadvacet let
si honem běžíš přivonět
a dřív než začneš vyprávět,
cos viděl - šeptáš ach jo.
Mívám chuť žárlit na ten květ,
to ovšem znáš už nazpaměť,
ať uschlé lístky vezme čert,
říkáš i bez nich, ach jo.
\endverse


\beginchorus
R: Budeš má, levandulová,
navždycky celá, celičká, levandulová,
famózně, nádherně levandulová,
hmm, levandulová.
\endchorus


\endsong
