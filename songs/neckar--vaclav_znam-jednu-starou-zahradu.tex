\beginsong{Znám jednu starou zahradu}[by={Neckář, Václav},
sr={},
cr={}]
\transpose{0}
% {{meta: source http://www.velkyzpevnik.cz/zpevnik/neckar-vaclav/znam-jednu-starou-zahradu}}


\beginverse
z pohádky "Šíleně smutná princezna"
Znám \[A]jednu starou zahradu,
kde \[Emi]hedvábná je tráva,
Má \[A]vrátka na pět západů
a \[Emi]mně se o ní zdává.
Tam \[D]žije krásná princezna,
má \[G]opá\[G4]  le\[G]nou \[D]pleť.
Jen \[Hmi]já vím, \[E]jak je \[G]líbe\[G7]zná,
tak \[C]neblázni a \[D]seď.
\endverse

\beginverse
\[G]Ná, ná, na-na-na-ná,
\[G7]na-na-na-ná, na-\[C]ná, ná.
\[B]Ná, ná, na-na-na-ná,
Es
na-na-na-ná, na-ná, ná.
\[G]Ná, ná, na-na-na-ná,
\[C]\[a-a-a-a-a-a-á]\[A]  \[á-a-a-á]  \[A7]\[a-a-á.]
\endverse

\beginverse
V té zahradě je náhodou
I studna s černou mříží.
A stará vrba nad vodou,
co v hladině se vzhlíží.
Ten rybník s loďkou dřevěnou
tu čeká na nás dva.
Tak pojď a hraj si s ozvěnou
a zpívej, to co já.
\endverse

\beginverse
Ná, ná...
\[a-a[C]-a-a-a\[Cmaj7]-a-a-a\[Ami]-á]
\[D7]\[a]\[g] \[-á.]
\endverse


\endsong
