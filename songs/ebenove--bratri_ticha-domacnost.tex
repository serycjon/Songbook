\beginsong{Tichá domácnost}[by={Ebenové, bratři},
sr={},
cr={}]
\transpose{0}
% {{meta: source http://www.velkyzpevnik.cz/zpevnik/ebenove-bratri/ticha-domacnost}}



\beginverse
1. N\[Hmi7]ení doma vžd\[E]ycky všechno t\[D/E]ak,
j\[Amaj7]ak by si člověk př\[Dmaj7]edstavoval,
n\[Hmi7]ěkdy to jde pr\[E]ávě naop\[D/E]ak,
s t\[Amaj7]ím bych vás nerad \[Dmaj7]unavoval,
\[Hmi7]u nás se nekř\[C#mi]ičí, \[F#mi]u nás se nesp\[C#mi]ílá,
\[Hmi7]u nás je zvláštní \[E]idyla.
\endverse

\beginchorus
R: T\[A]ichá, t\[D]ichá, n\[E]aše d\[Dmaj7]omácnost je t\[A/C#]ichá, t\[D]ichá
a s k\[E]onverzací n\[D]ikdo neposp\[A]íchá,
t\[D]ichá, n\[E]aše d\[Dmaj7]omácnost je t\[A/C#]ichá, t\[D]ichá,
jen v\[E]odovodní k\[D]ohout tiše vzd\[C#mi]ychá, t\[F#mi]ichá,
je p\[G]ozd\[D]ě h\[G]on\[D]it bych\[A]a.
\endchorus

\beginverse
2. Není doma vždycky všechno tak,
jak by to žena měla v plánu,
celý večer čekáte, a pak:
on přijde o půl páté k ránu,
u nás se nekřičí, u nás se nespílá,
u nás je zvláštní idyla.
\endverse

\beginchorus
R:\endchorus


\beginverse
3.=1.+2.
\endverse

\beginchorus
R:\endchorus




\endsong
