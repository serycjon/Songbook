\beginsong{Kravata č. 5}[by={Ebenové, bratři},
sr={},
cr={}]
\transpose{0}
% {{meta: source http://www.velkyzpevnik.cz/zpevnik/ebenove-bratri/kravata-c-5}}



\beginverse
1. N\[C]ež vyjdeš z domu\[F], n\[C]ež cv\[F]aknou vr\[C]átka\[F],
h\[C]oď j\[F]eště p\[C]ohled\[F], h\[C]oď d\[F]o zrc\[Bb]átka\[D#],
j\[Bb]estl\[D#]i máš tv\[Bb]rdý\[D#]  p\[Bb]uky\[D#] a r\[Bb]ysy\[D#],
j\[Bb]estl\[D#]i pod k\[Bb]rke\[D#]m t\[Bb]o, c\[D#]o má v\[Bb]iset, to, c\[D#]o má v\[Bb]iset, v\[A]is\[G#]í.
\endverse

\beginverse
2. Kravata tenká, ve vzorku zlato,
už nejsi v plenkách, už víš, jak na to,
kdo chce dokázat urvat pár frček,
musí si vázat, musí si svázat krček.
\endverse

\beginchorus
R: Protože sp\[G]olečenská smyčka, t\[C7]o je tvůj komplic,
z\[G]aškrtí slova, kt\[C7]erá jdou od plic,
\[G]ať se tvůj jazyk n\[C7]a chvíli postí,
vžd\[G]yť přece míříš d\[C7]o společnosti,
z\[G]avaž a ut\[Bb]áhn\[C]i, z\[G]avaž a ut\[Bb]áhn\[C]i,
z\[G]avaž a ut\[Bb]áhn\[C]i, z\[G]avaž a ut\[Bb]áhn\[C]i.
\endchorus

\beginverse
3. Zrcadlo žádám denně o radu,
jaký mám uzel dát si pod bradu,
zda zakřiknutý či agresivní,
inteligentní, anebo radši pivní.
\endverse

\beginverse
4. Do práce vážu, což není hloupé,
lodnický uzel - tam se to houpe,
až půjde o krk, mám ještě vždycky
v záloze uzlík, v záloze uzlík gordický.
\endverse

\beginchorus
R:\endchorus




\endsong
