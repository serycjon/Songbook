\beginsong{Archimedes}[by={Ebenové, bratři},
sr={},
cr={}]
\transpose{0}
% {{meta: source http://www.velkyzpevnik.cz/zpevnik/ebenove-bratri/archimedes}}



\beginverse
1. F\[D]ilosof\[C/D]em čl\[D]ověk se st\[C/D]ane,
j\[D]en když to d\[C/D]ělá f\[Hmi]orteln\[Ami7]ě,
mn\[D]ě vám to jd\[C/D]e  n\[D]ejlíp ve v\[C/D]aně,
mn\[D]ě to jde n\[C/D]ejlíp v k\[Hmi]oupeln\[Ami7]ě.
\endverse

\beginchorus
R: Kd\[C/D]yž vana př\[G]eték\[C]á, volám "h\[F]eurék\[C]a!"
a špl\[G]ouchám j\[Ami7]ako malý d\[D]ěcko,
kd\[C/D]yž vana př\[G]eték\[C]á, volám "h\[F]eurék\[C]a!",
a k\[G]oukat b\[Ami7]ude celý Ř\[D]ecko,
j\[Ami7]ak se dělá v\[Hmi7]ěda v\[Ami7]e vaně u \[C/D]Archiméd\[G]a.
\endchorus

\beginverse
2. Formulovat složité teze
je základ filosofie,
většinou chlap do vany leze
s tím, že se jenom umyje.
\endverse

\beginchorus
R:\endchorus


\beginchorus
R:\endchorus




\endsong
