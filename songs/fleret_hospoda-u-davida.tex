\beginsong{Hospoda U Davida}[by={Fleret},
sr={},
cr={}]
\transpose{0}
% {{meta: source http://www.velkyzpevnik.cz/zpevnik/fleret/hospoda-u-davida}}



\beginverse
1. \[C]V hospodě \[G]na náměstí \[C]můžete při \[G]troše štěstí
\[Ami]sehnat \[D]místo u sto\[G]lu,
sednout si \[C]na lavici, dát si \[G]pravou kyse\[Ami]lici
anebo \[D]rum a kofo\[G]lu.
\endverse


\beginverse
2. Říká se tam "U Davida", klasická třetí třída,
zákaz her s výjimkou domina,
a když je někdy pod psa venku, hospodský též dá si sklenku,
přisedne \[D]k vám a \[G]vzpomí\[C]ná.
\endverse


\beginchorus
R: \[H]Ač je to téměř k nevíře, já \[Emi]býval slavným kumštýřem,
\[H]zpíval a napříč flétnou \[Emi]hrál,
\[H]procestoval kraje cizí, \[Emi]v rozhlase a \[A]v televizi
\[C]každej mě \[D]znal,
\[H]a holky krásný jako břízky \[Emi]nosily mi z domu řízky
\[H]po koncertě rovnou do ša\[Emi]tny,
\[H]za podpisy do cancáku \[Emi]vlezly mi až \[A]do spacáku,
\[C]ale to je \[D]dnes už nepla\[G]tný.
\endchorus


\beginverse
3. Pak se napije a kývá hlavou, jako by přemýšlel nad zašlou slávou,
a je ticho a jen pendlovky jdou,
pak vstane a jde za své pípy, a za chvíli už zas vtipy
rozléhají se hospodou.
\endverse


\beginverse
4. A když nese další várku jak Děda Mráz s nůší dárků,
štamgastům se oči rozzáří,
co tam po odjezdu vlaků, hned si dají po panáku,
tržba se dnes jistě vydaří.
\endverse


\beginchorus
R: Jó, je to dnes až k nevíře, že ten chlap býval slavným kumštýřem,
zpíval a napříč flétnou hrál,
procestoval kraje cizí, v rozhlase a v televizi
každej ho znal,
a teď tady pivo točí a dojetím mu vlhnou oči,
když sem přijdou kluci s kytarou,
stojí v teskném zamyšlení a hosté mizí bez placení
a říkají si, že je zas pod párou.
\endchorus


\beginverse
5. Až někdy navštívíte Vizovice, zaparkujte u silnice,
tož tam, co je tech nejvíc obchodů,
rozhlédněte se vpředu, vzadu, a, nemáte-li oční vadu,
ucítíte hospodu.
\endverse


\beginverse
6. A když se zeptáte na hospodského, dovíte se od každého,
že je to notorický lhář,
ale já ho v\[C]iděl jednou v kv\[G/H]ětnu, jak tam st\[Ami]ál a v ruce fl\[D]étnu,
a kolem hl\[G]avy, kolem hlavy vám měl svatoz\[C]ář.
\endverse


\endsong
