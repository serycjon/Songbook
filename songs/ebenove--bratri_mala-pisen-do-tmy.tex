\beginsong{Malá píseň do tmy}[by={Ebenové, bratři},
sr={},
cr={}]
\transpose{0}
% {{meta: source http://www.velkyzpevnik.cz/zpevnik/ebenove-bratri/mala-pisen-do-tmy}}



\beginverse
1. M\[Bb]alá\[F] p\[Bb]íse\[F]ň d\[C]o tmy\[Ami] vžd\[Dmi]ycky př\[Emi]ijde vh\[Ami]od mi, š\[Dmi]etřím pr\[G]oud,
p\[Bb]íse\[F]ň sv\[Bb]ětl\[F]opl\[C]achá\[Ami]  šk\[Dmi]odu n\[Emi]enap\[G]áchá,
člověk ž\[Bb/D]asne, když se zh\[G]asne.
\endverse

\beginverse
2. Malá píseň do tmy dělá doprovod mi, když jsem sám,
píseň, co se stydí v přítomnosti lidí,
ta je tichá, nepospíchá.
\endverse

\beginchorus
R: Chc\[Dmi]eš-li si na něco p\[Emi]ořádně p\[Ami]osvítit, zhasni si,
chc\[Dmi]eš-li se něčeho \[Emi]opravdu n\[Ami]asytit, nejez\[G],
chc\[Dmi]eš-li své běžící m\[Emi]yšlenky z\[Ami]achytit, s\[F]edni si,
chc\[Ami]e to svůj čas, v\[Dmi7]ážně, chc\[G]e to svůj čas.
Rec: V této chvíli nevidím na ručičky hodinek,
když si zhasnete, taky na ně neuvidíte, to není marný,
v této chvíli nevidím na kalendář na zdi,
když si zhasnete, taky na něj neuvidíte, to není marný,
v této chvíli nevidím ani sám sebe,
když si zhasnete, taky se neuvidíte, to není marný,
k tomu všemu, ovšem nejenom k tomu,
se tedy hodí malá píseň do tmy.
\endchorus

\beginchorus
R:\endchorus


\beginverse
3. Každý, kdo si svítí, je zapojen v síti jako já,
kdyby stroje škytly, bude tma jak v pytli,
člověk zírá: černá díra.
\endverse

\beginverse
4. Pro případy nouze nehledejme dlouze světla zdroj,
mějme v každé době trochu světla v sobě,
aspoň škvírku, aspoň sirku ...
\endverse



\endsong
